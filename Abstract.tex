\section{``Executable Infrastructure for Collaboration at Regional
Level.''}\label{executable-infrastructure-for-collaboration-at-regional-level.}

\section{Abstract}\label{abstract}

E-Infrastructure has, over the years, proven its worth in enabling
scientific collaboration, even at regional and global scales. The
adoption of common platforms such as HPC, data management, etc has made
sharing of scientific data, applications and research outputs more
appealing and is accelerating scientific output, especially in regions
where these were previously unavailable - particularly Africa, and areas
of the Arab-speaking world. The CHAIN-REDS and ei4Africa projects have
supported the development of the Africa-Arabia Regional Operations
Centre (AAROC). The ROC acts as a coordination point, first for grid
infrastructures, but has been expanded to more general collaboration
infrastructure services such as science gateways, federated identity
providers, document and data repositories, etc. This expansion of
service offering to ever-more demanding research communities places
unreasonable strain on a fully-distributed model, where every site
administrator is expected to understand and operate these new services.
This is particularly true in the African and Arabian regions, where
knowledge networks are sparse.

This contribution describes a development and deployment philosophy
which adopts a ``DevOps'' paradigm which aims to encode models of
services using Ansible, with Github and Slack as collaboration
platforms. Site and service configuration has been coded into Ansible
playbooks, providing a reproducible model of the service, which can be
customised as desired on a per-site basis. Most importantly, this model
is executable, meaning that any number of sites and services can be
effectively deployed remotely, by a core team. Continuous integration is
done at every commit of code, by executing the playbooks on a
cloud-based development site, which provides transparency to the remote
site administrators.

The adoption of this methodology helps to solve the problems of
sustainably maintaining service configuration, improving communication
between site operations and service developers, ensuring the proper
state of services, and verifying the state of deployment.

Some of the main benefits of this approach are speeding up the
deployment of new services, reliably applying updates and recovering
from disaster. In this contribution, we show that the ``traditional''
HPC and grid service deployment can be reproduced and improved, by
adopting a more modern operations stack. However, we also highlight how
this has helped to deploy advanced services- federated identity
infrastructure, science gateways, application repositories, and Open
Access repositories in particular throughout the region in short time,
and is now playing a crucial role in the strengthening of technical and
scientific collaboration networks in the region.
